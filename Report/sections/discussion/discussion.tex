\chapter{Discussion}\label{ch:discussion}
Having created our final implementation of our model and tabu search, we will discuss both in this chapter. These will be general thoughts on what we have produced, including what we could have done differently.

\section{Modeling}\label{sec:modeling}
To rate configurations, which produce specific item orders, we developed a model in UPPAAL. With this model we can simulate items being produced by a factory. We may also generate the fastest timed trace, and from it extract the time it takes to produce the order. In this section, we will discuss what features we could have added to the model in addition to how it may be optimized further. We bookend the section by bringing up how the model compared to a real life factory. 

\subsection{Left Out Features}\label{ssec:LOFeatures}
In our current model, we may have a case, where the front element in a module's queue is waiting to be worked on by that module. At the same time, all items behind it may just need to pass through the module. On the actual CP Learning Factory setup, which we described in \cref{ssec:realcomparison}, they get around this by adding extra queues for items that need to be worked on by the module. In the case of the \textit{Robot Arm} module, it will guide items that it must work on onto an inner conveyor belt, where they will wait for processing. This ultimately allows for other items to more easily pass through the module. Thus we have a real life case, which our model is not designed to handle. The issue with implementing the feature is that it will increase the global state space during search, as fewer items are stuck in queues and are free to transition around the configuration. 

We were not able to exactly generate the running example from \cref{fig:running-example} through our tabu search. This is in part because we have not described a transformation rule, which allows us to parallelize modules serially as is the case with \textit{Jigsaw} and \textit{Buzzsaw} in the aforementioned example. We did not do this, as our model does not reap the benefits of such a configuration. In order to keep down the state space, our model requires that an item may not pass through a module, which may work on it. This means that any wooden sword we create will be sawed by \textit{Jigsaw} and then pass right through \textit{Buzzsaw}. The issue with removing this condition is that our search may unnecessarily branch further out into our configuration in search of another module, which may perform the same work.

Our model was not designed with our transformation rules in mind. We focused on keeping it as general as possible from the beginning. Yet we could get around the problem of parallelization increasing our state space, if we designed the model according to our parallelization rules. If we know that a serial parallelization, as with \textit{Jigsaw} and \textit{Buzzsaw}, can happen, then we know that we should only look ahead by one neighbour, to see if another module may perform the work. Thus we can limit the amount of lookahead branching performed during search.   

\subsection{Optimization}
On the issue of optimization in general, we know that our state space begins to grow fast as more branches are added to the configuration. This results in our tabu search becoming slower and slower, as we try to rate configurations of a rising complexity. As the search requires many configurations to be simulated, we want to keep the time spent rating each configuration to a minimum. We realize that there is a balance to find between the number of features, we add to the model, and the state space. We also risk a slower execution time, if the solutions that lower the state space increase the size of our states.

To optimize the model, we could choose to make it less general and implement optimizing features, based on how our transformation rules create new configurations. It may also be beneficial to partition a configuration into several sub-configurations. Rating each smaller part would not take as long as running the whole configuration, as the state space increases exponentially as our configuration becomes more complex. The issue here is that it becomes more difficult to model the real world, as we must concern ourselves with how to handle modules in sub-configurations passsing items onto modules in other sub-configurations. 

\subsection{Comparison to a Real Factory}
In \cref{ssec:realcomparison} we looked into how modeled configurations fared compared to a real factory. For simple orders our model fared well, but we see that when orders become more complex, then the total execution time of orders begins drifting apart. In our case, the drift may be because we have not modeled how items may block each other on the inside the modules. We observed that this happened when items had to leave the \textit{Robot Arm} module's work queue. We must also remember that we only compare the total production time of orders, not how the actual factory and modeled configuration fared against each other during production.

The factory, which we compared against, was very simple. If the factory had some of the features mentioned in \cref{ssec:LOFeatures} our results would have been different. In addition, it would be interesting to compare against a real factory, which had a branching production similar to our generated. In spite of this, we are rather pleased with our results, as they indicate that our model can be used for some real life configurations.



\section{Tabu Search}
To pick a good configuration for producing an order of items we used the meta-heuristic known as tabu search. In this section we will discuss our generation of neighbours and use of memory in the implementation.

\subsection{Generating Neighbours}
In our implementation of the tabu search, we based our neighbour generation functions on the form transformation rules that we sat up earlier. However these functions are very different in how many neighbours we generate. Swap especially tends to generate a lot of different neighbours. Instead of generating all possible neighbours, it would instead be interesting to stochastically generate neighbourhoods. By just generating a subset of neighbours which are picked stochastically, we could equalize the  amount of neighbours that we evaluate at each iteration. Thus more of the total search time may be spend on interesting backtracking using long term memory, which may be more beneficial to the search.

Each of the three neighbourhood functions has a certain probability of being picked. We looked into having these probabilities altered as more search iterations are performed. Thus changing the search rules. Our basic implementation made it very likely to do anti-serializations first, but as time goes on, parallelization and swap become more likely. The idea being that anti-serialization sets up some basic configuration structure, which the others can then alter in less dramatic ways. It would be interesting to look more into, how we may change up our search rules, and when this should happen.

\subsection{Using Memory}
We discover that short term memory is not used that often during search. This is because our transformation rules has a very low chance of later generating the same configurationn. This may only occasionally happen when we generate neighbours using swap. Short term memory is sometimes used, when we backtrack using long term memory. Yet only if we backtrack to a recent element in long term memory, where it's earlier best neighbour is still in short term memory.

Long term could also be used in a more interesting way. Right now it has unlimited size and every frontier picked is stored within it. We also only use it to backtrack in the case that we find no better frontier. A possibility of backtracking using long term, which increases at each iteration, could allow us to backtrack more and search new areas. 

It could be interesting to define a memory not as a configuration, but as a set of attribute values. Thus we would be able to catch more unwanted neighbours using the short term memory. The issue here being how to meaningfully describe a configuration only by a few attributes, which can be used to guide search. 



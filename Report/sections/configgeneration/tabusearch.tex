\section{Tabu Search}
We have now defined, how we may set up configurations and orders in python, as well as how to get the rating of a configuration producing a specific order. We now tackle the problem of finding the optimal configuration, given an order of items and a set of avaliable modules. Finding the optimal configuration can be a difficult problem to solve. Therefore we do not focus on creating an optimization algorithm, but rather a heuristic. Unlike optimization algorithms, heuristics do not guarentee a globally optimal solution. Instead it promises a locally optimal solution, which may be good enough in practice.

\subsection{Choosing a Metaheuristic}
A metaheuristic is a problem-independent technique used to develop heuristics for optimization problems. We choose to look into the local search family of metaheuristics. These describe heuristics, where we try to tackle the problem of optimizing some measure, by moving between candidate solutions to some problem. In our case the measure we want to optimize is the configuration rating, where we want to find the configuration with the minimal value.

The most basic form of local search is hill climbing. Here we perform a local search by starting from some initial candidate solution. We then generate the neighbouring solutions and compare them on their measure value. The neighbour with the best measure is chosen as the new frontier. Search continues by generating all neighbours to this solution as to compare measures and find an even better frontier. This continues until we have a frontier, where no neighbours have a better measure. This last frontier is then chosen as the best solution. The problem with this approach is that we are very likely to move into a local optima. 

Tabu Search is another type of local search developed by Fred W. Glover \cref{glover2006}. It focuses on using memory as a way of moving away from local optima that catches us, when hill climbing. Memory comes in two different flavors short term and long term. As we search, any solution that we pick as frontier is added to the short term memory, also known as a tabu list. Any solution reciding in short term memory can not be chosen as frontier. This allows us to pick neighbours, which we would otherwise have dismissed as frontiers. In addition we are allowed to pick an ill fit neighbour to be frontier, if no other neighbour optimizes our measure. Both of these constructs guide us around local optima.

Long term memory has a few different definitions. It may be used as an extension of short term memory. Both short and long term have a limited size. This means that they will have to forget old frontiers to make room for new ones. When short term memory forgets a solution, it may end up in long term memory and continue to have us steer clear of that particular solution. In other implementations long term memory is instead used to reset search. It may happen that we enter a bad search area with many low ranking local optima. To escape this, we may replace our current frontier with one found in long term memory, as to drive search into a new area.

Memories may also be defined in different ways. A memory may describe an entire solution or perhaps just a subset of attribute values for a solution. In short term memory, the later will in general exclude more possible frontiers. 

When performing a tabu search there needs to be a balance between diversification and intensivation. Hill Climbing is pure intensivations, where we constantly try to optimize our measure. Random search on the other hand is pure diversification, where there is no active attempt made to optimize our measure. Depending on implementation, many factors in a tabu search may control diversification and intensivation. As an example, the larger short term memory is, the more diversification. The smaller it is the more intensivation.


\subsection{Tabu Search Implementation}


\begin{minipage}{\linewidth}
\lstinputlisting[language=Python, caption= pseudocode showing a simplified version of the tabu search implementation, captionpos = b, label={code:psuedotabu}]{codeRelated/Python/psuedo_tabu.py}
\end{minipage}

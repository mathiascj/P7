\chapter{Metoder}\label{ch:met}
I dette kapitel vil vi redegøre for hvilke UX metoder vi har valgt at bruge i vores evaluering af Codecademy. Vi har valgt at begge metoder skal være meget forskellige fra hinanden, så de kan give os forskellige perspektiver. Begge er fundet igennem \textit{All About UX}\cite{AllAboutUX}.


\section{Affect Grid}\label{sec:AG}
Vi vil gerne vide, hvordan brugeres oplevelse ændrer sig under interaktion med Codecademy. Vi har altså brug for en metode, der kan anvendes til at indsamle data flere gange under vores evaluering. 

Under evaluering kan det dog være svært, løbende at sætte sig ind i en bruger føler. Hvis evaluatoren hele tiden stiller dybdegående spørgsmål under evalueringen, risikerer vi at brugeren bliver trukket ud af interaktionen. Dette kommer til at have indflydelse på brugerens oplevelse af produktet. Det vil altså være bedst hvis brugen selv kunne angive, hvordan de føler under evalueringen. Helst uden at de skal tænke for meget over det. 

Metoden vi vælger at anvende hedder Affect Grid. Den er udviklet inden for psykologien af James A. Russel samt Anna Weiss og Gerald A. Mendelsohn \cite{AffectGrid}. Her angiver brugeren selv deres følelsesmæssige tilstand, også kendt som \textit{affect}, på det såkaldte Affect Grid. Et blankt affect grid kan ses på \cref{fig:affectgrid}. Grid'et har en størrelse på 9*9 felter, X-aksen angiver fornøjelse(pleasure) og Y-aksen angiver ophidselse(arousal). Et 1 på X-aksen betyder stærk misfornøjelse, mens et 9 betyder stærk fornøjelse. Ligeledes betyder 1 på Y-aksen stærk søvnighed og 9 betyder stærk ophidelse. Hele metoden baserer sig på, at fornøjelse og ophidelse er uafhængige af hinanden. 

\begin{figure}[h]
\centering
\includegraphics[width=0.6\textwidth]{affectgridbase.png}
\caption{Blank Affect Grid}
\label{fig:affectgrid}
\end{figure}

Ved anvendelse af dette grid kan en bruger nemt angive \textit{affect} ved at overveje deres nuværende fornøjelses- og ophidelsesniveau og derefter markere på gridet. Da dette ikke kræver meget af brugeren, regner vi med at der kan markeres flere gange under evalueringen, uden at brugeren trækkes ud af interaktionen. Derfor anvender vi Affect Grid til løbende at indsamle data om brugeres \textit{affect} under evalueringen.


\section{3E}\label{sec:3E}
Foruden den kvantitative data fra \textit{Affect Grid} ønsker vi ogspå at indsamle kvalitativ data om brugernes oplevelse. Det kan dog være svært at indsamle kvalitativ data løbende under selve testprocessen, da det typisk tager længere tid at for en brugere at levere denne form for data. Dette kan medføre at brugeren bliver revet ud af sin oplevelse af systemet. 

Vi har derfor valgt at søge efter en metode som kan bruges på et episode plan, dvs. efter at en testperson har interegeret med systemet og som kan fange personens oplevelse. Vi vælger da at bruge metoden \textit{Expressing Experiences and Emotions} (\textit{3E}. Her får testpersonen udleveret et stykke papir, hvorpå der er tegnet en tændstiksmand, en talebobel og en tænkebobel. Et eksempel kan ses på \cref{fig:3E}. De bliver da bedt op at tegne på papiret med henblik på at projektere deres oplevelser og følelser. Dette lader personen udtrykke sig på en nem og useriøs måde. Tændstikmandens formål er at opstille nogle rammer for tegningen.  \textit{3E} er videreudviklet fra \textit{Emocards}-metoden, hvor brugeren udtrykker sig ved at vælge fra en håndfuld tegnede ansigter. Med den blanke tændstikmand er det tanken, at brugeren har større mulighed for at udtrykke sig. 


\begin{figure}[h]
\centering
\includegraphics[width=0.6\textwidth]{3EBlank.jpg}
\caption{Blank 3E}
\label{fig:3E}
\end{figure}


Et problem med \textit{3E} er at det er svært at analysere dataen bagefter, da den indsamlede data kan fortolkes på mange måder. Derfor bedes testpersonen om at forklare deres tegning under eller efter udførsel.  
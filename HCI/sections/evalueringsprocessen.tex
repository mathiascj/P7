\chapter{Testprocessen}\label{ch:evaproc}
I dette kapitel redegør vi for, hvordan de overnævnte metoder anvendes i testprocessen. Vi har testet fire personer individuelt. Alle er mandlige studerende på datalogi. En evaluator guider testpersonen under testudførsel, mens en anden gør notat. En trin for trin gennemgang af testprocessen, som hver testperson føres igennem, kan ses nedenfor:

\begin{itemize}
\item Introduktion og beskrivelse af \textit{Affect Grid}.
\item Opvarmningsrunde med \textit{Affect Grid}.
\item Testpersonen stilles løbende en række interaktionsopgaver, der skal løses i Codecademy. Efter udførsel af hver opgave bedes personen om at markere på et blankt affect grid.
\item Når testpersonen har løst alle opgaver introduceres \textit{3E} med eksempel.
\item Testpersonen tegner en \textit{3E} tegning under tænke højt.
\end{itemize}

For at vi kan bruge en metode skal en testperson have kendskab til den, samt vide hvordan han selv indgår i den. Derfor bruger vi tid, på at gøre hver testperson vant med begge metoder.  
Udover verbalt at forklare testpersoner om \textit{Affect Grid} får vi dem til at lave en opvarmningsrunde før der interageres med Codecademy. Her fremvises der fem tilfældige billeder, og for hver bedes testpersonen om at markere på et blankt affect grid. Formålet er at gøre dem vante med metoden, så de hurtigt kan markere under den egentlige evaluering. Denne introduktion tager ca. fem minutter.

Under interaktionen med Codecademy bliver testpersonerne løbende stillet opgaver, der leder dem til at logge ind på siden, melde sig på et kursus, gennemføre en lektion og derefter logge ud igen. Der markeres på affect grid undervejs, hovedsageligt under lektionen.
Da alle testpersoner har en datalogisk baggrund, vælger vi at de skal udføre en lektion i et sprog, som de ingen kendskab har til, Ruby. Dette gøres for at give testpersonerne et mere ens udgangspunkt for interaktionen. Ruby vælges, da det er et meget simpelt sprog at forstå. Derpå undgår vi at interaktionen går i stå, fordi en testperson skal bruge lang tid på at opfange et nyt begreb. Vi vælger at testpersonerne skal springe den introducerende Ruby lektion over, da den er meget simpel. I stedet skal de under interaktionen udføre lektion nummer to i Ruby kurset, da denne præsenterer en større problemstilling, spredt over flere opgaver. Vi erfarer at hele interaktionen med 11 interaktionsopgaver i alt tager 15 til 20 minutter. 

Efter at en testperson har logget ud af Codecademy, som det sidste led i interaktionen, introduceres \textit{3E}. Forklaringen foregår verbalt og understøttes af et eksempel på en tegning.  Her har vi tegnet et ansigt på vores tændstikmand, for at vise hvordan personen kan udtrykke sig.  Imens personen tegner beder vi dem også om at tænke højt. Dette gøres med henblik på lettere at kunne tolke tegningen efterfølgende. Dette tager ca. 5 minutter i alt. 

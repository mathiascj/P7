\chapter{Introduction}
\label{ch:introduction}
We stand in the middle of a new paradigm shift within manufacturing known as Industry 4.0. While the term has no official definition, the European Parliamentary Research Service (EPRS)\cite{Davis2015} defines it as entailing the transformation of traditional manufacture using modern digital technologies. In part, these include wireless communication, cyber-physical systems, simulation and modular factory systems.

Modular factory systems are made up of individual production modules, each performing some kind of job in a production line. These systems are flexible as their modules can be rearranged into a new production line. This is opposed to more traditional lines, which do not change often after being installed. The added flexibility of these systems allows manufacturers to quickly change the product, which they manufacture, to meet fast-changing market demands. The production of small product batches also becomes more affordable. This is beneficial when trying to reach niche markets, or when producing product prototypes as part of an iterative development process.

However, with these benefits comes the challenge of constantly re-planning the design of production lines. Traditional manufacturing concerns such as maximizing throughput of course have to be considered. Yet, we must also take care of challenges unique to the modular factory, such as the cost of moving from one production line to the next. This leads us to the project's initial problem:

\bigskip
\textit{How may it be possible to optimize the throughput of individual production lines in a modular factory system, while keeping down the overall cost of production?}
\bigskip

We will narrow down this problem in the following problem analysis, as to arrive at a specific goal for this project. 


\chapter{Introduktion}\label{ch:intro}
Inden for HCI har man tidligere fokuseret på et produkts pragmatiske kvalitet. Det vil sige, dets evne til at lade brugeren opfylde \textit{do-goals}, hvilke beskriver hvad brugeren ønsker at gøre. Denne tilgang munder ud i \textit{Usability} begrebet. God \textit{usability} handler om at et system er effiktivt, nemt og sikkert at bruge. Computersystemer bliver altså tænkt på som værktøjer, der skal bruges til at løse opgaver.

En anden tilgang kaldes User Experience \textit{UX}. Her fokuseres der på produktets hedoniske kvalitet frem for dets pragmatiske. Dette betyder at vi i stedet kigger på produktets evne til at opfylde \textit{be-goals}. I \cite{Hassenzahl:2008} beskriver Hazzenzahl disse som værende menneskelige behov en bruger ønsker at opfylde så som autonomi, stimulering og popularitet.
Hazzenzahl beskriver UX som en momentær følelse opnået igennem interaktionen med et produkt. Ydermere, så kan opfyldelsen af \textit{do-goals} hjælpe os med at opfylde \textit{be-goals}. For eksempel, så kunne et \textit{do-goal} for det gamle Pacman være at gennemføre et niveau. Hvis dette er muligt, kunne det understøtte et \textit{be-goal}, der omhandler en positiv stimulering som følge af sejr. Det er dog muligt, at vores \textit{be-goals} ikke opfyldes selvom vores \textit{do-goals} gør. Et system er altså mere en bare summen af  dets enkelte dele. Derfor siger vi at UX tager en holistisk tilgang til produktevaluering. Der fokuseres på brugerens oplevelse med systemet, hvilket er det endelige udfald af alle de små designbeslutninger der ellers er blevet taget.  

Definitionen af UX er dog stadigvæk debatteret. Samtidig er mange UX dimensioner blevet bragt op i litteraturen. I \cite{Bang2015} kategoriserer Bang disse dimensioner som relaterende sig til enten generel UX, indtryk, brug eller kontekst. Mange forskellige metoder er også blevet fremsat til at evaluere UX. 

I denne rapport vil vi forsøge at anvende to UX metoder \textit{Affect Grid} og \textit{3E} til at lave en summativ evaluering af hjemmesiden Codecademy \cite{Codecademy}. Hjemmesiden tilbyder diverse online kurser, der tages med henblik på at lære sig programmeringssprog. Fokuset ligger dog på at evaluere metoderne, og ikke Codecademy i sig selv.
 
Rapporten er struktureret på følgende måde: I \cref{ch:met} redegøres der for vores valgte metoder. \cref{ch:evaproc} vil forklare vores brug af metoderne under en evaluering af Codecademy. \cref{ch:evalmet} evaluerer de to metoder. I \cref{ch:konk} konkluderes der på rapporten.



   
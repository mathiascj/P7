\chapter{Evaluering af Metoder}\label{ch:evalmet}
I dette kapitel vil vi evaluere \textit{Affect Grid} og \textit{3E} metoderne. Her kigger vi nærmere på den indsamlede data, og giver udtryk for vores egen oplevelse af metoderne. Ifølge Woolrych i \cite{Woolrych} er det vigtigt at forstå en metode som en mængde af konfigurerbare ressourcer. Ofte er en metode blot en vag opskrift, der giver os en eller flere ressourcer at arbejde med. For både \textit{Affect Grid} og \textit{3E} er vi kun blevet stillet med en måde, hvorpå vi kan indsamle data. Vi har selv besluttet, hvordan opgaver udvælges, hvordan data analyseres osv. På baggrund af dette kan vi ikke lave en endelig dom over metoderne. Vi kan dog beskrive, hvordan vi mener de forskellige ressourcer burde udvælges og konfigureres.

\section{Affect Grid}\label{sec:evalAG}
Med 11 markeringer fra alle 4 personer ender \textit{Affect Grid} med at indsamle 44 datapunkter. På \cref{tab:AG} har vi forsøgt at analysere dette datasæt med nogle standard statistiske metrikker. Vi ser at testpersoner i gennemsnit føler sig neutrale ift. både fornøjelse og ophidselse. Dog er der en høj standardafvigelse, hvilket kræver dybere undersøgelse.  

\begin{table}[]
\centering
\caption{Statistiske metrikker over Affect Grid data}
\label{tab:AG}
\begin{tabular}{|l|l|c|}
\hline
                  & Fornøjelse & Ophidelses               \\ \hline
Antal             & 44         & 44                       \\ \hline
Gennemsnit        & 5.3        & 5.7                      \\ \hline
Standardafvigelse & 2.0        & 2.2                      \\ \hline
Min               & 1.0        & 1.0 \\ \hline
Max               & 1.0        & 1.0 \\ \hline
\end{tabular}
\end{table}

På \cref{fig:hexbin} har vi plottet dataene på et hexbin diagram. Dette viser udspredelse i langt større grad. Selvom vi ser at mange punkter er placeret neutralt omkring midten, så bliver alle hjørner af diagrammet berørt. Ved at kigge nærmere på hvordan testpersoner har markeret, ser vi også en stor forskel i, hvordan enkelte opgaver opleves. Dette kan ses i \cref{app:AG}, hvor et søjlediagram over markeringer er lavet for hver testperson. Generelt ser vi dog nogle større negative udsving, når testpersoner er færdige med opgaver som de har haft problemer med. Testpersonerne bliver da ikke nødvendigvis stimulerede af at overkomme en svær udfordring. 

\begin{figure}[h]
\centering
\includegraphics[width=0.6\textwidth]{hexbin.png}
\caption{Hexbin diagram over Affect Grid data}
\label{fig:hexbin}
\end{figure}

Samlet set viser dataene at testpersonerne har oplevet interaktionen med Codecademy på meget forskellige måder. Fra at have observeret personerne under evalueringen regner vi dog ikke med, at forskellen skulle være så stor. To faktorer vi tænker kan have haft en indflydelse på forskellen er; testpersonernes initielle sindelag samt deres forståelse af metoden.
I \cref{app:AG} observerer vi at det første kryds sat af testpersonerne afviger meget fra hinanden. Da de endnu ikke har interageret med systemet på dette punkt, regner vi ellers med at de markerer i samme område. Dette tyder på at testpersonerne har haft forskellige mentale udgangspunkter, hvilket har haft en indflydelse på deres oplevelse. Dette gør det svært at sammenligne data fra enkelte personer. Vi regner dog med at dette problem kan løses ved at indsamle data fra flere personer. Dog kunne det være interessant at kigge nærmere på, hvordan data kunne normaliseres ud fra en persons initielle sindelag.
Det er også muligt at hver testperson har haft en forskellig forståelse af metoden. Testperson 3 er for eksempel mere tilbøjelig til at hoppe ud i ekstremer end Testperson 2. Dette kunne tyde på en forskel i hvordan ekstremer opfattes. Vi observer også deciderede forståelsesproblemer, hvor en person spørger ind til modellen under evaluering. Det lader altså til, at vi skulle have anvendt en anden teknik til at forklare modellen. En verbal ad hoc forklaring, som vi gav, giver ikke nok forståelse. 

\textit{Affect Grid} blev originalt valgt, da den kunne bruges til at indsamle data under evaluering, uden at trække testpersonen ud af interaktionen. Vi finder dog at testpersonerne stadigvæk bruger meget tid på at overveje, hvor deres krydser skal sættes. Dette kan relatere sig til forståelsesproblemer som nævnt overfor. Det er problematisk, da testpersonerne bliver trukket ud af interaktionen. De endte muligvis også med at skulle markere for ofte. Der skal være en balance, hvor vi får nok data uden at testpersonen forstyrres. 
Alt i alt ser vi et potentielle i \textit{Affect Grid}. Dog kræver metoden en større indsats fra vores side for få testpersonerne til at forstå den. Dette med henblik på at de de får den samme opfattelse og hurtigere kan markere på grid'et. Samtidig må der kræves en mere struktureret teknik til at finde ud af, hvornår der skal markeres under evaluering.  

\section{3E}\label{sec:eval3E}
Med fire forsøgspersoner fik vi fire tegninger ud af \textit{3E}, som kan ses i \cref{bilag}. Med lidt løs fortolkning kan man ud fra tegningerne antage at testpersonernes oplevelse var positiv, begrundet af de glade ansigter. Den eneste person som ikke tegnede et glad angsigt var Testperson 2. Testpersonerne var tilbøjelige til at snakke om usability problemer, mens de tegnede. Det var dog kun Testperson 4 som viste sin irretation overfor dette på sin tegning. 
Generelt har vi haft mange problemer med \textit{3E}. Vi føler at vores testpersoner havde svært ved at udtrykke deres følelser grafisk og på en sådan måde, at vi kan drage nogen konklusion fra det. Dette kan til dels skyldes metodens placering i testprocessen. Testpersonerne skulle anvende \textit{3E} lige efter at have arbejdet med programmeringsopgaver. Disse to opgavetyper er meget forskellige fra hinanden, og testpersonerne burde måske have fået mere tid til at indstille sig på en ny type opgave.   

Samtidig var det nok en fejl at bede personerne om at tænke højt under tegning, da de ofte endte med at stoppe med at tegne og i stedet snakke med os. Det er også muligt at testpersonerne følte sig pressede af at vi kiggede på, mens de tegnede. Mange er ikke vant til at tegne overfor andre, og slet ikke under pres. Vi burde i stedet have ladet dem tegne i enerum og derefter spurgt ind til tegningen. 

Det er også muligt at folk havde svært ved at udtrykke sig, fordi Codecademy ikke rigtig inspirerede nogle stærke følelser. Uden inspiration havde personerne svært ved at finde på noget at tegne. Vi har derfor også eksempler på tilfældige illustrationer, så som en drage der spyder ild hos Testperson 4. 

Vi tænker ikke at vi vil anvende \textit{3E} til at teste lignende systemer. Det vil dog være interessant at anvende metoden på systemer, der er mere stimulerende. I vores tilfælde kunne vi have fået den samme data ud af at bruge \textit{Emocards}, da testpersonerne endte med at udtrykke sig med simple påtegnede ansigtsudtryk.



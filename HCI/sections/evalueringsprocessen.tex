\chapter{Testprocessen}\label{ch:evaproc}

I dette kapitel vil hvordan de to metoder blev brugt under testprocessen blive redegjort. Selve testprocessen vil blive gennemgået med fokus på de to metoder. En simpleficeret version af testprocessen kan ses nedenfor.

\begin{itemize}
\item Introduktion og beskrivelse af \textit{Affect Grid}.
\item Opvarmningsrunde med \textit{Affect Grid}.
\item Testpersonen bliver stillet opgaver løbende som han skal løse i Codecademy.
\item Efter hver opgave bliver testpersoner bedt om at markere sin oplevelse på et affect grid.
\item Når testpersonen har løst alle opgaver bliv han introduceret til \textit{3E} samt hvist et eksempel.
\item Testpersonen tegner så en \textit{3E} tegning imens at han verbalt forklare hvorfor han tegner det han gør.
\end{itemize}

Introduktionerne til begger metoder gøres pågrund af at det antages at testpersonerne ikke har tidliger erfaringer med metoder og hvordan de skal interagere med dem. Vi forklare derfor deres rolle i metoden og hvordan vi forventer at de skal bruge dem.

Da grunden til at vi valgte \textit{Affect Grid} var at vi ville have data punkter fra undervejs testprocessen og samtidigt ikke vil hive testpersonerne for langt væk fra deres interaction med systemet, valgte vi at lave en opvarmningsrunden til \textit{Affect Grid}. Det gjordes med den tanke at hvis brugeren allerede havde prøvet sig af med \textit{Affect Grid} forinden de egentlige opgaver, så vil han bedre være istand til hurtigt at udfylde fremtidige affect grids med lille konsekvens på hans iteraction med codecademy.

Da alle testpersonerne har datalogisk baggrund, valgte vi at de skulle udføre opgaver i codecademy som fokuserede på at lære et sprog som alle fire testpersoner ikke havde bekendskap med før, nemlig Ruby. Vi valgte også at de ikke skulle begynde helt fra starten i Ruby kurset, da de første mente at de første opgaver var for simple og ville kede brugerne for meget. Et problem med at få vores testpersoner til at udførre opgave i et nyt sprog er dog at de nogle gange helere vil evaluere sproget som de er i gang med at lære end selve siden hvori i lærer sproget.

Efter alle opgaver var løst bad introducerede vi testpersonen til \textit{3E}. Dette blev gjort hovedsageligt ved at vise dem et eksempel og forklare dem hvad vi forventede af dem. Eksempel var også ment som at være en kilde for inspiration, så som at vi havde tegnet et glad ansigt på tændstikmanden og at dette skulle hjælpe med at vise brugeren at de også kunne udtrykke dem selv med at tegne ansigter på manden. 

Imens at testpersonen tegnede hans bad vi ham forklare hvad han tegenede undervejs og hvorfor han tegnede det. Dette blev gjort så vi ikke behøves tolke på hvad han har tegnet senere og i tilfælde af at vi ikke tyde hvad det var han havde tegnet.  
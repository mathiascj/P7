\pdfbookmark[0]{English title page}{label:titlepage_en}
\aautitlepage{%
  \englishprojectinfo{
    Optimizing Modular Factory Configurations %title
  }{%
    From reality to models %theme
  }{%
     Fall semester 2016 %project period
  }{%
    d701d16 % project group
  }{%
    %list of group members
    Alexander Brandborg \\
    Mathias Claus Jensen
  }{%
    %list of supervisors
    Søren Enevoldsen
  }{%
    0 % number of printed copies
  }{%
    \today % date of completion
  }%
}{%department and address
  \textbf{Department of Computer Science}\\
  Selma lagerløfs vej 300\\
  \href{http://www.cs.aau.dk/}{http://www.cs.aau.dk/}
}{% the abstract
  Formålet med denne rapport er at udvikle et system, hvormed vi kan genere valide og effiktive fabrikskonfigurationer baseret på en mængde løse fabriksmoduler samt en ordre bestående af produkter, der skal produceres. Vi udvikler en model I UPPAAL, der gør at vi kan simulere produktionen af ordre på en konfiguration. I forlængelse kan vi bruge denne til at vurdere, hvor hurtig en konfiguration er til at producere sin ordre.  Herefter definerer vi formelle regler, der bruges til at generere valide fabrikskonfigurationer fra et simpelt udgangspunkt. Disse regler implementeres i python og bruges, i tandem med vores UPPAAL model, i en implementeret tabu search, hvis formål er at søge konfigurationer igennem og finde den hurtigste.  Ved sammenligning med en menneskeskabt konfiguration, finder vi at vi kan genere en ikke ens, men alligevel fornuftig konfiguration til at producere en ordre. 
}



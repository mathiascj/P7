\chapter{Konklusion}\label{ch:konk}
I denne rapport har vi beskrevet vores UX evaluering af Codecademy. Vi valgte to metoder til evalueringen. \textit{Affect Grid} til at indsamle kvantitativ data under interaktion. \textit{3E} til at indsamle kvalitativ data efter interaktion. Under vores testprocess lod vi fire testpersoner individuelt løse Ruby opgaver i Codecademy. De overnævnte metoder blev brugt til at indsamle data om hvordan hver testperson oplevede interaktionen. Ud fra metoderne kan vi sige at brugerne havde en positiv, dog ikke meget stimulerende oplevelse af Codecademy. \textit{Affect Grid} endte med at være den metode vi fik mest ud af, da den gav os en dybere forståelse for terspersonernes oplevelse undervejs. Dog vil vi notere at metoden kræver at brugerne har en rigtig god forståelse af hvordan den skal bruges. Samtidig skal er tænkes meget over hvor ofte man skal have testpersoner til at markere på grid'et. \textit{3E} gav os ikke særlig meget brubar data, da testpersonerne havde svært ved at formulere sig selv visuelt. Vi tænker til dels at dette var på grund af at Codecademy ikke er et system der skaber de stærke føleleser. Vi mener da at \textit{3E} i stedet burde bruges til systemer der er mere følelsesmæssig involverende/stimulerende. Til systemer som Codecademy vil vi for fremtiden fortrække \textit{Emocards}. 


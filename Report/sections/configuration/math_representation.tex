\section{Mathematical Representation}

In this section we will describe factory configurations, a long with the context of the factory using set theory. Meaning that we will describe how the different modules are connected in the configuration, what modules are free, and the recipes that the configuration is supposed to be making. This is done so that we later can mathematically explain our rules for finding neighbours in our tabu search.

Previously we presented recipes as acyclic dependency graphs, as this level of abstraction is not required for reconfiguring our configuration, we will simply define a recipes as the set of all work that is required to be done by a recipe.

\[recipe: \textit{A set of all work required to be done by a concrete recipe}\]

\noindent We will also define modules as the following tuble:

\[module: \textit{(aW, mW, up, right, down, left, shadow)}\]

\noindent Where each element of the tuple represents the following:
\begin{itemize}
\item $aW$: The set of work that the module is actively doing.
\item $mW$: The set of work that the module can do.
\item $up,\, right,\, down,\, left$: Another module or empty.
\end{itemize}

The modules defined here closely resembles our previous definition of a module, with the addition of $aW$, which is later used to simplify the logic of some our rules.

Furthermore we define something called a line. A line is a linear ordering of modules. The ordering of the modules is done by the $\prec_r$ operand, where $a \prec_r b$ is true if we from $a$ can traverse using only $right$ attributes to get to $b$.

\[line: \textit{A total ordered set of modules ordered on } \prec_r\]

We now define a configuration as the following tuple:

\[configuration: (R, M, \Gamma)\]

Where:
\[R: \textit{The set of all recipes}\]
\[M: \textit{Set of all modules}\]
\[\Gamma: \textit{The set of all lines}\]



Furthermore as each $module$ in a $line$ should not be present in another $line$, we say that the following is true for each $line$ $\gamma$, in $\Gamma$:
\[\texttt{if } \gamma \in \Gamma \texttt{ then } \forall m \in \gamma \land \forall \gamma ' \in \Gamma \land \gamma \neq \gamma ',\, m \notin \gamma ' \]

We also say that the following is true each $module$ $m$ in $M$.
\[\texttt{if } m \in M \texttt{ then } m.aW \subseteq m.mW \land m.mW \subseteq  \bigcup_{r\in R}r\] 


Besides these sets, we describe some derived sets that will later be used for when we present our rules. First we describe the set of all modules used by the configuration, denoted $CM$, and the set of all free modules, denoted $FM$:


\[CM = \{m \in \gamma | \gamma \in \Gamma \}\]
\[FM = M \setminus CM \]

Given a specific recipe $r$, we define a special operation, $\bar{r}$, as as the union of the work that all recipes other than $r$ can do. 
\[\bar{r} = \bigcup_{r' \in R}r', \texttt{ if } r' \neq r\]
This is used for finding the set all modules within a line $\gamma$ where the recipe $r$ and some other recipe are being worked upon, denoted $K_{\gamma ,r}$. The set of all modules within $\gamma$ where $r$ isn't being worked upon, denoted $\alpha_{\gamma ,r}$. And the set of all modules within $\gamma$ where only $r$ is being worked on, denoted $beta_{\gamma ,r}$. 
\[K_{\gamma ,r} = \{m \in \gamma | \gamma \in \Gamma \land \exists \rho \in m.aW,\, \{\rho\} \subseteq r \land \{\rho\} \subseteq \bar{r} \land r \in R\}\]

\[\alpha_{\gamma ,r}  = \{m \in \gamma | \gamma \in \Gamma \land \forall \rho \in m.aW,\, \{\rho\} \nsubseteq r \land r \in R\}\]

\[\beta_{\gamma ,r}  = \{m \in \gamma | \gamma \in \Gamma \land \forall \rho \in m.aW,\, \{\rho\} \subseteq r \land \{\rho\} \nsubseteq \bar{r} \land r \in R\}\]


We define a set $KP_{\gamma ,r}$, such that it contains all pairs $(s, e)$ of $K_{\gamma ,r}$ modules where we can transverse from $s$ to $e$ without there being another $K_{\gamma, r}$ modules on the way.
\[KP_{\gamma ,r} = \{(s, e)| {s, e} \in K_{\gamma ,r} \times K_{\gamma ,r} \land s <_k  e\}\]
Where $s <_k e$ means that we can traverse right from $s$ to $e$ without seeing another $k_{\gamma, r}$ module. 

We define an operation, $M_{s,e}$, that given a pair $(s, e)$, where $(s, e) \in P_{\gamma ,r}$, we get all modules inbetween $s$ and $e$.
\[M_{s,e} = \{m | m \in \gamma \land \gamma \in \Gamma \land s \prec m \land m \prec e\}\]

We then define similar rules for finding all $\alpha$ and $\beta$ modules inbetween $s$ and $e$.
\[A_{s,e} = \{m |m \in M_{s,e} \land m \in \alpha\}\]
\[B_{s,e} = \{m |m \in M_{s,e} \land m \in \beta\}\]


\[PP_{s, e} = \{(m, m')| m \in M_{s,e} \land m' \in FM \land m.aW \subseteq m'.mW\} \]

\[ PP_{options} = \{p \in {PP}_{s,e}^2 | (m,m') \in p \land |p| = |M_{s,e}| \land  \forall m , (n,n') \in p \land m \neq n \} \]


\[P_{s, e} = \{m'| m \in M_{s,e} \land m' \in FM \land m.aW \subseteq m'.mW\} \]

\[ P_{options} = \{p \in {P}_{s,e}^2 | (m,m') \in p \land |p| = |M_{s,e}| \land  \forall m , (n,n') \in p \land m \neq n \} \]




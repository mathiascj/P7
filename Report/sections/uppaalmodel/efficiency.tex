\section{Efficiency}\label{subs:efficiency}
In \cref{subs:recipequeue} we illustrate how we make search for the best timed trace more efficient, by ordering how recipes are placed onto a factory. By applying such order we are trying to keep down the global state space like how the bourgeois tries to keep down the proletariat.   

This is not the only part of the model, in which we have tried to make our search more efficient. We have strayed from going deeper into it until now, as it does not impact the overall functionality of the model. Yet it does greatly impact runtime. 

The use of urgent channels and committed locations are what mainly speeds up search.

Committed locations ensure that globally no action may be taken and no time may pass before leaving a location. This ensures that some processes stay atomic, such as the handshake and work on recipes described in \cref{subs:recipe}. This guards us from the interference of other instance, but also greatly reduces the search space.

Urgency is an attribute that can be applied to different channels. Having a channel be urgent means that no time may pass, if we are in a state, where we can synchronize on the channel. In \emph{ModuleWorker} on \cref{subs:moduleworker} the intern channel is urgent. This means that when the guards evaluate to true, then we will synchronize on \emph{intern} as soon as possible. Not allowing the system the possibility of waiting again reduces the state space. There are transitions we wish to make urgent, such as from \emph{Done} to \emph{Working}, where they are not already communicating on a channel. We solve this by creating a new simple template called \emph{Urgent}. An instance of this will continuously try to synchronize on the urgent \emph{urg} channel. Thus we can add a synchronization on this channel to any non-urgent transition to make it urgent. 

A big difficulty with working with committed locations and urgent channels is that we sometimes need to be able to wait. This is the case when passing from \emph{ModuleWorker} to \emph{ModuleTransporter} through the intern channel as mentioned above. We can not safely make \emph{intern} urgent, if we may synchronize on it from \emph{ModuleWorker} but not   \emph{ModuleTransporter}. This will lead to unwanted deadlock.  Therefore, the guard on this particular transition states that it must only be taken, if the \emph{ModuleTransporter} is in idle and can communicate. We use this technique throughout the model, applying guards so that we may safely make our channels urgent. 

In order to shave more off the state space, we have also created priorities between channels and between instances using a built-in UPPAAL feature. In the case where we may communicate on either channel or transition on either instance, the priorities tell us which should be done. With less choice the state space shrinks.

A feature which UPPAAL lacks is a user-defined constructor, which would allow it to execute code right when an instance is made. We need this feature in templates such as \emph{ModuleQueue} in \cref{subs:modulequeue}, where we need to set up the array, which implements the queue. To get around this, we add an initial location to the template. To transition from this location, the instantiating functions have to be called. The issue here is that with N instances having to be instantiated, there are N! sequences of performing instantiation. Again, this greatly increases our state space. To get around this, we use a queueing system similar to \emph{RecipeQueue} in \cref{subs:recipequeue}. This is implemented with the \emph{Initializer} template as seen in \cref{fig:initializer}. Each instance, which needs to have code run upon instantiation, is given an instantiation id. When instantiating the \emph{Initializer}, we give it an array, which orders all these ids. The \emph{Initializer} then starts to instantiate the instances in the order given to it by synchronizing on the \emph{initialize} channel. As the \emph{Initializer}'s main location is committed, nothing may occur until all needed instances have been synchronized with. Once all instantiation has been done, the \emph{Intanializer} is forced to move into a dead state and we need no worry about it again.    

\begin{figure}[h]
\centering
\includegraphics[width=\textwidth]{init.pdf}
\caption{The Initialzier template}
\label{fig:initializer}
\end{figure}


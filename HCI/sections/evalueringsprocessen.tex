\chapter{Testprocessen}\label{ch:evaproc}
I dette kapitel redegør vi for, hvordan de overnævnte metoder bruges under testprocessen. Vi endte med at teste 4 personer individuelt. Alle mandlige studerende på datalogi. Under test var der en evaluator der guidede testprocessen, og  en anden der gjorde notat.  En step by step gennemgang af testprocessen kan ses nedenfor:

\begin{itemize}
\item Introduktion og beskrivelse af \textit{Affect Grid}.
\item Opvarmningsrunde med \textit{Affect Grid}.
\item Testpersonen bliver stillet opgaver løbende som han skal løse i Codecademy.
\item Efter hver opgave bliver testpersonen bedt om at markere sin oplevelse på et blankt affect grid.
\item Når testpersonen har løst alle opgaver bliv han introduceret til \textit{3E} samt fremvist et eksempel.
\item Testpersonen tegner så en \textit{3E} tegning, mens han verbalt forklarer tegningen.
\end{itemize}


For at vi kan bruge en metode skal en testperson have kendskab til den, samt vide hvordan han selv indgår i den. Derfor bruger vi tid på at gøre hver testperson vant med hver metode.  

For at introducere testpersoner til \textit{Affect Grid} fik vi dem til at lave en opvarmningsrunde inden interaktion med Codecademy. Her blev de vist 5 tilfælde billeder og skulle for hver markere på et blankt affect grid. Formålet var at gøre dem vante med metoden og evt. snakke med dem om uklarheder i vores formelle forklaring. På baggrund af dette burde de være i stand til hurtigt at markere på et affect grid når de bedes om det. 

Da alle testpersonerne har en datalogisk baggrund, valgte vi at de skulle udføre opgaver i codecademy, som fokuserede på at lære et sprog, Ruby, som ingen af dem havde kendskab til. Ruby blev valgt, da det er et meget simpelt sprog at forstå, så alle testpersoner når at lave de samme opgaver. En af udfordringerne ved at evaluere Codecademy er at man let kan komme til at evaluere sproget og ikke siden. Vi valgte at testpersonerne skulle springe de introducerende Ruby opgaver over, da de var meget simple, og i stedet udføre en større afgrænset opgave. Efter udførsel af hver delopgave bedes brugeren om at markere på et blankt affect grid. 

Efter alle Codecademy opgaver er løst, introduceres testpersonen til \textit{3E}. Dette gøres ved en formel forklaring af metoden samt en forklaring af hvad der forventes af personen. Vi forsøgte også at forklare  ved brug af et eksempel. Her havde vi tegnet et ansigt på tændstikmanden for at vise at hvordan personen kunne udtrykke sig.  Imens personen tegnede bad vi dem også om at tænke højt. Dette gøres med henblik på lettere at kunne tolke tegningen efterfølgende. 

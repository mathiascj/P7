\section{Formal transformation rules} \label{sec:math_rules}
In this section we describe how we may transform a factory configuration, initially made up of a single line of modules that can complete all recipes, to another configuration, which may be more efficient. First we will formally describe what a configuration is. Later we describe rules, used to transform one configuration into another.

\begin{definition}[htb]
$recipe: \textit{Set of all work required to complete a concrete recipe.}$
\\ \\
$module: \textit{Set of all work that a module may perform on an item}$
\\ \\
$line: \textit{Set of modules totally ordered on the relation} \prec$
\\
\caption{Basic elements in our definitions}
\label{def:basic}
\end{definition}


In \cref{def:basic} we define some basic elements, which make up a configuration. We treat a recipe, a specefic type of item, as a set of works, since our initial configuration will always be made up of a single line of ordered modules able to complete all given recipes. And as none of the transformation rules described later will change this order, we do not need to enforce an order on the works that make up a recipe. A module is described as the set of works that it may perform on an item. A line is a set of modules totally ordered on $\prec$. The relation orders modules according to their placement in the line from left to right. If $m_1 \prec m_2$ for modules $m_1$ and $m_2$ in a line, then there is a horizontal path from $m_1$ to $m_2$ traveling rightwards. 

\begin{definition}[htb]
$Configuration: (R, M , \Gamma, \Gamma_0, Activeworks, Start, End)$
\\
Where: \\
$R:$ A set of recipes \\
$M:$ A set of modules \\
$\Gamma:$ A set of lines \\
$\Gamma_0:$ Main line \\
$ActiveWorks: M \rightarrow \bigcup_{r\in R}r$ \\ 
$Start: \Gamma \rightharpoonup M$ \\
$End: \Gamma \rightharpoonup M$ \\
\caption{Formal definition of a configuration}
\label{def:config}
\end{definition}


In \cref{def:config}, we have defined a factory configuration as tuple made up sets and functions. $R$ and $M$ are never changed by our transformation rules. Thus they constant, going from an initial configuration forwards. As stated before, each configuration starts as a single initial line. For the first configuration, $\Gamma_0$ represents this line. In subsequent configurations $\Gamma_0$ represents a special line from which a subset of our transformations will allow us to branch out from.

$ActiveWorks$ is a function, which for a module $m \in M$ returns the set of works, which $m$ performs activly on at least one recipe $r \in R$. $Start$ is a partial function, which maps a line $\gamma \in \Gamma$ to a module $m \in M$ such that $m in \gamma'$ where $\gamma' \in \Gamma$ and $\gamma \neq \gamma'$. This represents that there is a vertical path between $m$ and the first element of $\gamma$. Similarly the partial function $End$, which maps a line $\gamma \in \Gamma$ to a module $m \in M$ such that $m in \gamma'$ where $\gamma' \in \Gamma$ and $\gamma \neq \gamma'$. This represent that there is a vertical path from the last element of $\gamma$ to $m$.   

Furthermore we only want each $module$ to appear in a single $line$. To enforce this we set up the following rule for each $line$ $\gamma$, in $\Gamma$:
\[\texttt{if } \gamma \in \Gamma \texttt{ then } \forall m \in \gamma \land \forall \gamma ' \in \Gamma \land \gamma \neq \gamma ',\, m \notin \gamma ' \]

We also set up the following rule for each $module$ $m$ in $M$. This is done to ensure that the module can not perform work not within a recipe, and that the work which it actively performs should be a part of its total set of works:
\[\texttt{if } m \in M \texttt{ then } ActiveWorks(m) \subseteq m.mW \land m.mW \subseteq  \bigcup_{r\in R}r\] 

\subsection{Anti-serialization}
In this section we will describe a set of transformation rules known as anti-serializations. On \cref{fig:trivial-example} all items made according to the three recipes in \cref{fig:toy-recipes} will have to pass through modules, which do not work on them. We would like to minimize this as to reduce the total length that each item must be transported, in addition to combating bottlenecks on the line. Thus it would be beneficial if items could bypass modules that do not work on them. We get around this by branching out production on more lines, where items only go through modules that work on them. 

\subsubsection{Branching between common modules}
We start by defining the most common type of anti-serialization with the $AS_0$ transformation rule given in \cref{def:as0}. Informally the rule states that if we have a set of modules $B_{r,s,e}$, which works the recipe $r \in R$ exclusivly, lying between the modules $s$ and $e$ on $\Gamma_0$ then we may branch out all modules in $B_{r,s,e}$ to form a new line. This new line is connected to $\Gamma_0$ at $s$ and $e$. We will explain this more in depth, by going over the sets brough up in the rule. 

\begin{definition}[htb]
\runatt{AS}{0}
    \infrule
        {0 < |B_{r,s,e}| \land  s <_k e}
        {(R,M,\Gamma, \Gamma_0, AW, Start, End) \rightarrow_{AS}
        (R,M,\Gamma', \Gamma_0', AW, Start', End') } \\
        Where: \\
        $r \in R$ \\
		$s,e \in K_{\Gamma_0,r}$\\		
		$\Gamma_0' = (Pre_s \cup \{s\}  \cup A_{r,s,e} \cup \{e\} \cup Pos_e, \prec)$ \\     
        $\Gamma' = (\Gamma \setminus \{\Gamma_0\}) \cup \{\Gamma_0'\} \cup B_{r,s,e} $ \\
		$Start' = Start \cup \{(B_{r,s,e}, s)\}$ \\
		$End' = End \cup \{(B_{r,s,e}, e)\}$

\caption{Formal definition of the $AS_0$ transformation rule}
\label{def:as0}
\end{definition}

We initially define a special set on $r$, $\bar{r}$, which contains all types of work which are not unique to $r$:
\[\bar{r} = \bigcup_{r' \in R}r', \texttt{ if } r' \neq r\]

Next we define, what we refer to as common modules. These are modules in $\Gamma_0$, which $r$ needs to use along with at least some other recipe. These are important to identify, as they may not be branched out, as that would make them inaccessible to the other recipes than $r$.  Given $r$ and $\Gamma_0$ we define the set of common modules $K_{\Gamma_0,r}$ as follows:

\[K_{\Gamma_0 ,r} = \{m | m \in \Gamma_0  \land \exists \rho \in AW(m),\, \{\rho\} \subseteq r \land \{\rho\} \subseteq \bar{r} \land r \in R\}\]

From this set we can then define $\alpha_{\Gamma_0 ,r}$, which is the set of modules in $\Gamma_0$ which are not used by $r$: 

\[\alpha_{\Gamma_0 ,r}  = \{m |m \in \Gamma_0 \land \forall \rho \in AW(m),\, \{\rho\} \nsubseteq r \land r \in R\}\]

Along with $\beta_{\Gamma_0 ,r}$, which is the set of modules in $\Gamma_0$ used only by $r$:

\[\beta_{\Gamma_0 ,r}  = \{m  | m \in \Gamma_0 \land \forall \rho \in AW(m),\, \{\rho\} \subseteq r \land \{\rho\} \nsubseteq \bar{r} \land r \in R\}\]

Next we define the binary relation relation $<_K$ as:
\[a <_K b = \left\{\begin{matrix}
tt \texttt{ if } a,b,c \in K_{\Gamma_0 ,r} \land a \prec b \land \lnot (a \prec c \land c \prec b) \\ \texttt{else } ff
\end{matrix}\right.\]

We also define a total order of all modules in between two modules, $s$ and $e$, in some line $\gamma \in \Gamma$:

\[M_{s,e} = \({m | m \in \gamma \land \gamma \in \Gamma \land s \prec m \land m \prec e\}, \prec)\]

We can now define the total order of all modules appearing between $s$ and $e$, which are not used to work $r$, $A_{r,s,e}$ as follows: 

\[A_{r,s,e} = (\{m |m \in M_{s,e} \land m \in \alpha\}, \prec)\]

Similarly we can define the total order $B_{r,s,e}$, which contains all modules that appear between $s$ and $e$ which work exclusivly on $r$

\[B_{r,s,e} = (\{m |m \in M_{s,e} \land m \in \beta\}, \prec)\]

Lastly we define $Pre_{s}$ which is the set of all modules that come before a module $s$ in line $\gamma \in \Gamma$:
\[Pre_{s} = \{m | m \in \gamma \land \gamma \in \Gamma \land m \prec s\}\]

In addition to $Pos_{e}$ which is the set of all modules that come after a module $e$  in line $\gamma \in \Gamma$:
\[Pos_{e} = \{m | m \in \gamma \land \gamma \in \Gamma \land e \prec  m \}\]

Having defined these sets the rule in \cref{def:as0}, shows that in the case where $|B_{r,s,e}| > 0 \land s <_K e$ then we may update $\Gamma$ with two new lines, including an update of $\Gamma_0$ that describes a branch out from $\Gamma_0'$, which exclusivly works upon $r$ before rejoining with the old line.   

While this rule is formally sound it can be difficult to read. As such we have tried to illustrate the transformation from configuration to configuration visually in \cref{fig:as0}. This visualization uses a homemade notation. Square boxes refer to individual modules. Boxes with rounded corners are total orders of sets of modules. A box with rounded corners containing "..." means that here is a total order of one or more modules, but we do not care what they contain. A graphical line of boxes flowing from left to right is a line in$ \Gamma$ ordered as shown. If two lines $\gamma$ and $\gamma'$ are connected by a vertical upward going edge, then $Start$ contains the ordered pair made up of the module $s$ in $\gamma$ directly below the start of $\gamma'$ and the first element of $\gamma'$. At the same time, if the edge goes downwards from $\gamma'$ to $\gamma$ then $End$ contains the ordered pair including the module $e$ in $\gamma$, which is directly below the end of $\gamma'$ and the last element of $\gamma'$. 

\begin{figure}[H]
\centering
\includegraphics[width=\textwidth]{as0.pdf}
\caption{A visual representation of the $AS_0$ transformation rule}
\label{fig:as0}
\end{figure}

Having now defined our sets, we will set up the rules describing how we may branch out modules to work on individual recipes. These rules will be described informally using our own graphical notation. The graphical notation works as follows. Square boxes represent modules. Boxes with rounded corners represent a set of modules totally ordered on $\prec$. The modules within the box are connected left to right according to their order. Connections going to the box connect to the first element of the order and connections leaving from the box come from the last module. If a set box only contains "..." it means that some set of modules may be here, but there is no need to consider what it contains. 

Now imagine the situation in \cref{fig:asbase}. Here we have specific $s$ and $e$ modules chosen on a line $\gamma$ and we have chosen to branch out a specific recipe $r$. Between the two modules we may have $M_{s,e}$ placed. From this we can calculate $A_{s,e}$ and $B_{s,e}$ as described before. If $B_{s,e} = \emptyset$ then we may branch out some modules only used to work on $r$. 

\begin{figure}[H]
\centering
\includegraphics[width=\textwidth]{as0.pdf}
\caption{A visual representation of the $AS_0$ transformation rule}
\label{fig:as0}
\end{figure}

If $|A_{s,e}| = |B_{s,e}| + 2$ , then we can branch out as shown in the top of \cref{fig:astrans}. Here we simply remove $|B_{s,e}|$ from the rest of $M_{s,e}$ leaving us with in its place $A_{s,e}$. The module $s$ is then connected upwards to the first element of $|B_{s,e}|$. The last element of $|B_{s,e}|$ is then connected to $e$. This transformation also entails that all of  $|B_{s,e}|$ is removed from $\gamma$ and that a new line containing $|B_{s,e}|$ is added to $\Gamma$. For each of the transformations presented in this subsection we alter $\gamma$ and add to $\Gamma$ the new branch which is created.  Notice that the modules beneath the new line are marked as red. This means that the shadow variable in the module tuples on the old line have been set to true. This is used in order to handle transformation conflicts as described later in \cref{ssec:tc}.

In the middle of \cref{fig:astrans} we describe the case where $A_{s,e}| > |B_{s,e}| + 2$. The difference between the cardinality of these two sets is called i.  In this case we can not physically connect the last element of $|B_{s,e}|$ downwards to $e$. We get around this issue by appending the new line with i new transport modules. These are modules which can not perform any work and are only used to transport recipes. The last of these transporters is then connected downwards to $e$. The new line added to $\Gamma$ consists of the modules in $|B_{s,e}|$ ordered behind the new transport modules as depicted in the figure. 

The last case, shown in the bottom of \cref{fig:astrans} is similar to the previous one, but instead $A_{s,e}| < |B_{s,e}| + 2$. In this case we append the original line with transporters instead. This requires an update of $\gamma$ to include these new transporters, while the new line added to $\Gamma$ is again just $|B_{s,e}|$. 

\begin{figure}[H]
\centering
\includegraphics[width=\textwidth]{as2.pdf}
\caption{3 different configurations to which we may go from \cref{fig:asbase}. Top: Case when $A_{s,e}| = |B_{s,e}| + 2$. Middle: Case when $A_{s,e}| > |B_{s,e}| + 2$. Bottom: Case when $A_{s,e}| < |B_{s,e}| + 2$ }
\label{fig:astrans}
\end{figure}

In the rest of this sub-section we will describe two special cases known as branch in and branch out. 

\subsubsection{Early Branch In}\label{sssec:bi}
It may be that we have modules that precede the first module of $K_{\gamma ,r}$, known as $K_{first}$. We define a new operand to get all modules appearing before a module $s$ in its line:

\[Pre_{s} = \{m | m \in \gamma \land \gamma \in \Gamma \land m \prec s\}\]

Using this with $K_{first}$, we get the set of modules preceding $K_{first}$ known as $Pre_{K_{first}}$. We can then define the set $B_{first}$, containing modules used only by $r$ in $Pre_{K_{first}}$ as:

\[ B_{first} = \{m | m \in \beta_{\gamma ,r}  \land m \in Pre_{K_{first}} \} \]

Similarly the set of all modules preceding $K_{first}$, which do not exclusively work on $r$ as: 

\[ A_{first} = \{m | m \in \alpha_{\gamma ,r}  \land m \in Pre_{K_{first}} \} \]

With this in place we present the transformation rule in \cref{fig:asbranchin}. This describes how we in the above case are allowed to remove $B_{first}$ from the line $\gamma$ leaving  $A_{first}$. We then connect its last module to $K_{first}$. The new line which is added to $\Gamma$ is then $B_{first}$. This type of branch allows us to start $r$ away from other items.   


\begin{figure}[H]
\centering
\includegraphics[width=\textwidth]{as4.pdf}
\caption{The case where modules precede the first module in $K_{\gamma ,r}$}
\label{fig:asbranchin}
\end{figure}

\subsubsection{Final Branch Out}
We have a similar case to \cref{sssec:bi}, where the last element in $K_{\gamma ,r}$, known as $K_{last}$, is proceeded by a set of modules. To get the modules proceeding $K_{last}$ we define the operand $Pro_{e}$ as follows:

\[Pro_{e} = \{m | m \in \gamma \land \gamma \in \Gamma \land e \prec  m \}\]

Using this on $K_{last}$ gives us $Pro_{K_{last}}$. The set containing all modules proceeding $K_{last}$, which exclusively work on $r$, is known as $B_{last}$ and is defined as:

\[B_{last} = \{m | m \in \beta_{\gamma ,r}  \land m \in Pro_{K_{last}} \}\]

The set of all modules proceeding $K_{last}$, which do not exclusively work on $r$, $A_{last}$ is given by:

\[ A_{last} = \{m | m \in \alpha_{\gamma ,r}  \land m \in Pro_{K_{last}} \} \]

Having set up this we present the final transformation rule for this type of branching in \cref{fig:asbranchout}. Here we see that in the described case we may remove $B_{last}$ from the line $\gamma$ leaving  $A_{last}$. $K_{last}$ is then connected to the first module of $B_{last}$. The new line added to $\Gamma$ is $B_{last}$. This transformation allows us to branch out $r$ when there is no need for it to stick with the old line never to return.

\begin{figure}[H]
\centering
\includegraphics[width=\textwidth]{as3.pdf}
\caption{The case where modules proceed the last module in $K_{\gamma ,r}$}
\label{fig:asbranchout}
\end{figure}

\subsubsection{Restricting branches}
For the previous transformation rules, we have imagined that no other branch has been made from our line to any side. Yet, this will not always be the case. To be flexible enough to produce configurations with high throughput, we need to handle this. We go more in depth with how to enforce some general physical rules in \cref{ssec:conflicts}, but here we will tackle one unique to anti-serialization.

On the top part of \cref{fig:shadowexample}, we see that we have already made a branch from $s1$ to $e1$, which results in the modules on the old being shadowed. Being shadowed means that if we remove the module and just reconnect the old line as usual, then the branch becomes too long to connect back to its old line. The line needs to connect back to its designated point on the old line as the module located here is a common module. It performs work on the recipe $r$ otherwise worked on by the new line and should therefore not be bypassed.  To get around this we, as shown in \cref{fig:shadowexample}, alter our anti-serialization transformation in this case to replace a shadowed module with a transport module, as not to skew the two previous lines away from each other. The example shows this done for a single module, but it may be done regardless of the amount of shadowed modules which we remove.

We may however not do this if the shadowed module is also a module where a branch either connects from or to. This means that the module is common and is used by the branch. Removing it would keep us from producing that branch's specific recipe. Therefore we never allow for these modules to be removed, even if this means that we can not perform certain anti-serializations.

\begin{figure}[H]
\centering
\includegraphics[width=\textwidth]{as5.pdf}
\caption{Transformation that handles the case where an anti-serialization removes a shadowed module}
\label{fig:shadowexample}
\end{figure}

 
\subsection{Parallelization}
In this section we will describe a set of transformation rules known as Parallelizations. In \cref{fig:trivial-example} we use only the exact number of modules necessary to produce the three recipes given in \cref{fig:toy-recipes}. It could easily be imagined that we had more modules capable of doing the same work as the ones in \cref{fig:trivial-example}. As such we produce a set of transformation rules that allow for possible greater throughput by parallelizing free modules with similar modules already in a line.


\subsubsection{$Para_0$: Parallelization Between Common Modules}
We start by defining the most common type of parallelization with the $Para_0$ transformation rule. Informally the rule states that if we have the total order $M_{s,e}$ of modules in between the modules $s$ and $e$. Then we can add a parallel line starting at $s$ and ending at $e$, if we can find an existing total order of free modules $P_{s,e}$ that can perform the same work as the modules in $M_{s,e}$. The visual representation of the $Para_0$ transformation rule can be seen \cref{fig:para0}. Here we expand the visual representation of our rules with the square box with a T in it. This box is just represents a single module that can perform no work, commonly referred to as a transport module.

\begin{figure}[H]
	\centering
	\includegraphics[width=0.7\textwidth]{para0.pdf}
	\caption{A visual representation of the $Para_0$ transformation rule}
	\label{fig:para0}
\end{figure}

In order to find the total order, $P_{s,e,}$, that can perform the same work as $M_{s,e}$. We first define the set of all free modules, $FM$. We say that the set of all free modules is the relative complement of all modules and all modules in a line $\gamma \in \Gamma$.

\[FM = M \setminus \{m | m \in \gamma \land \gamma \in \Gamma\}\]

We then describe a set of pairs of free modules and modules in between $s$ and $e$, where the free module $m$ can at least do the same work as the module in between $s$ and $e$ is currently doing.

\[ParaMap_{s, e} = \{(m, m')| m \in FM \land m' \in M_{s,e} \land AW(m') \subseteq m\} \]

We then describe all sets of pairs that could be a possible parallel line for the modules between $s$ and $e$. Note that this set could be the empty set, as the modules needed for creating a possible line might not available in $FM$.

\[ParaMapPaths_{s,e} = \{p \in ParaMap_{s,e}^2 | (m,m') \in p \land (n,n') \in p \land |p| = |M_{s,e}| \land  \forall m': m' \neq n' \}\]

After this we define $s[1]$ as the operation that given a set of pairs $s$, gives the set of all the first elements of the pairs in $s$.
\[s[1] = \{m_1 | (m_1, m_2) \in s\}\]

Using this we define the total order $P_{s,e}$ as:

\[ P_{s,e} = (p[1], \prec), \texttt{ where } p \in ParaMapPaths_{s,e} \]

This definition means that there might not exist any $P_{s,e}$ at all, or that there might be multiple candidates for $P_{s,e}$.

\subsubsection{$Para_1$: Parallelization without forking module}

\begin{figure}[H]
	\centering
	\includegraphics[width=0.8\textwidth]{para1.pdf}
	\caption{A visual representation of the $Para_1$ transformation rule}
	\label{fig:para1}
\end{figure}

\subsubsection{$Para_2$: Parallelization without joining module}


We also have two other cases, one in which we have no end module, meaning that we will fork out a parallel line, and the case where we have no start, meaning we will join in a parallel line. A graph showing both of these cases can be seen respectively in \cref{fig:para_we} and \cref{fig:para_ws}. For the case without start we connect the last module in the total order $(P_{s,e}, <_p)$ to $e$, and for the case without end we connect $s$ to the first module the in the total order $(P_{s,e}, <_p)$.


\begin{figure}[h]
\centering
\includegraphics[width=0.3\textwidth]{para_we.pdf}
\caption{Parallelization rule when you do not have an end module}
\label{fig:para_we}
\end{figure}

\begin{figure}[h]
\centering
\includegraphics[width=0.3\textwidth]{para_ws.pdf}
\caption{Parallelization rule when you do not have a start module}
\label{fig:para_ws}
\end{figure}

 
\subsection{Swap}
It would also be smart if we could swap out modules in the configuration with free modules, or two modules within a configuration that is capable of doing each others active work.

For this we describe two sets, $Swap_{extern}$ and $Swap_{intern}$. These sets contain all pairs of modules which we can swap with each other. A graph showing each of the swap rules can be seen respectively in \cref{fig:swap_ext} and \cref{fig:swap_int}.

\[Swap_{extern} = \{(m, m') | m \in CM \land m' \in FM \land m.aW \subseteq m'.mW \}\]

\[Swap_{intern} = \{(m, m') | m, m' \in CM \land m.aW \subseteq m'.mW \land m'.aW \subseteq m.mW \}\]

\begin{figure}[h]
\centering
\includegraphics[width=0.8\textwidth]{swap_ext.pdf}
\caption{External swap rule. Here $a$ denotes the first element in a pair from $Swap_{extern}$ and $b$ denotes the second element}
\label{fig:swap_ext}
\end{figure}

\begin{figure}[h]
\centering
\includegraphics[width=0.8\textwidth]{swap_int.pdf}
\caption{Internal swap rule. Here $a$ denotes the first element in a pair from $Swap_{intern}$ and $b$ denotes the second element}
\label{fig:swap_int}
\end{figure}

For the extern case in \cref{fig:swap_ext} we simply swap out the first element of a pair in $Swap_{extern}$ with the second element of the pair. Assigning the $up$, $right$, $down$, $left$ and $aW$ attributes of the second element equal to that of the first. We also remove the first element from the line it was in and insert the second in its stead, with the same total ordering.

We do something similar for the intern case in \cref{fig:swap_ext}. Here we instead of removing one of the two element in the pair, we assign the $up$, $right$, $down$, $left$ and $aW$ attributes of both elements to that of the other, and swap them in their respective lines.

We also need to describe that not all available modules need to be a part of our factory configuration. There may exist free modules, which are not used, but may aid in the search for a higher throughput. First we describe the set of all modules used by a configuration, denoted $CM$:

%\[CM = \{m \in \gamma | \gamma \in \Gamma \}\]
%The set of free modules $FM$ is then given by:
%\[FM = M \setminus CM \]
 
\section{Conflicts}\label{ssec:conflicts}
As mentioned back in \cref{ch:uppaalmodel}, we did not enforce the physical rules of modules to the fullest. These entail that a module may not connect to another module if this module is not a neighbour and if the connection would force two modules to take up the same space. The anti-serialization transformations rules can currently create configurations in which not all modules are connected to neighbours, and this problem will be solved in \cref{ssec:restrictbranch}. Furthermore both our anti-serialization and parallelization rules can cause intersections to occur, and we will solve this problem in \cref{ssec:paround} and \cref{ssec:pbeneath}.


\subsection{Restricting Branches Anti-Serialization}\label{ssec:restrictbranch}
For the anti-serialization transformation rules described in \cref{sec:as} we did not make sure that when we connected new lines to the main line, that the start and end of these lines were neighbours to the first and last module of the line. Furthermore, untill now, we have imagined that no other branch has been made from our line when performing the transformation. Yet, this will not always be the case, and as such can cause some conflicts.

\subsubsection{Neighbour Errors}
Imagine a situation similar to what the rule $AS_0$ in \cref{fig:as0} describes. Here we have specific $s$ and $e$ modules chosen on a line $\gamma$ and we have chosen to branch out a specific recipe $r$. Between the two modules we have $M_{s,e}$. From this we can calculate $A_{r,s,e}$ and $B_{r,s,e}$ as described before. If $0 < |B_{r,s,e}|$, then we may branch out some modules only used to work on $r$. However in  most cases $B_{r,s,e}$ will not have a size such that it can be connected in reality to the $s$ and $e$ modules. We therefor modify our $AS_0$ transformation rule so that solves this.

If $|A_{r,s,e}| + 2 > |B_{r,s,e}|$ , then the result of our $AS_0$ transformation rule will be the top of \cref{fig:astrans}. In this case we append the new line with transport modules to make the two lines fit each other. Notice that the modules beneath the new line are marked as shadowed, i.e. coloured red. This is used in order to handle transformation conflicts as described below in Shadowed Modules. We also introduce blue rounded boxes with indexes to the top right of them. These boxes mean that anything inside of them are serialized $i$ times, where $i$ is the index given with the box. 


If $|A_{r,s,e}| + 2 \leq |B_{r,s,e}|$,  then the result of our $AS_0$ transformation rule will be the bottom of \cref{fig:astrans}, In this case we append the original line with transporters instead of the new line to make them fit together.

\begin{figure}[H]
	\centering
	\includegraphics[width=\textwidth]{astrans.pdf}
	\caption{The new results of $AS_0$}
	\label{fig:astrans}
\end{figure}

\subsubsection{Shadowed Modules}
On the top part of \cref{fig:shadowexample}, we see that we have already made a branch from $s1$ to $e1$, which results in the modules on the other line being shadowed. Being shadowed means that if we remove the module and just reconnect the old line as usual, then the branch becomes too long to connect back to its old line, and shadowed boxes are visualized with the colour red. The line needs to connect back to its designated point on the old line as the module located here is a common module. It performs work on the recipe $r$ otherwise worked on by the new line and should therefore not be bypassed.  To get around this we, as shown in \cref{fig:shadowexample}, alter our anti-serialization transformation in this case to replace a shadowed module with a transport module, as not to skew the two previous lines away from each other. The example shows this done for a single module, but it may be done regardless of the amount of shadowed modules which we remove. In \cref{fig:shadowexample} we also introduce rounded boxes with three dots in them. These boxes just means any number of modules in a total order.

We may however not do this if the shadowed module is also a module where a branch either connects from or to. This means that the module is common and is used by the branch. Removing it would keep us from producing that branch's specific recipe. Therefore we never allow for these modules to be removed, even if this means that we can not perform certain anti-serializations.

\begin{figure}[H]
	\centering
	\includegraphics[width=\textwidth]{as5.pdf}
	\caption{Transformation that handles the case where an anti-serialization removes a shadowed module}
	\label{fig:shadowexample}
\end{figure}

\subsection{Push Around} \label{ssec:paround}
In push around we handle the case where a new off branching line intersects with an old line, by moving the new line above the old line. In the case where the new line entirely covers the old we perform the transformation depicted in \cref{fig:pusharound1}. As shown, the new line simply uses transport modules to lift itself up above the old line. These transport modules are technically not a part of the line and only serve the purpose of avoiding the intersection.

\begin{figure}[h]
\centering
\includegraphics[width=\textwidth]{conflict1.pdf}
\caption{How push around handles the case, where we insert a new line that covers the entirety of an old line}
\label{fig:pusharound1}
\end{figure}

There is also the case where the new line is covered by the old line entirely. In this case we use the transformation in \cref{fig:pusharound2}. Here we do not need to append any transport module as we move up through the already existing modules, which we would otherwise intersect with. 

In the cases where the new entirely intersects the old it will add transport modules where needed, or otherwise guide itself vertically through modules that already exists. These examples only show the case, where a single old line is placed above the main line from which we branch off. In the case of more line levels, the new line will simply climb vertically until it finds a level where it may be placed without intersection. 

Push around is the intersect handling we use when inserting new lines as a result of anti-serialization, as there is no need to keep this new line close to the one from which it sprouted. 

\begin{figure}[H]
\centering
\includegraphics[width=\textwidth]{conflict2.pdf}
\caption{How push around handles the case where we insert a new line that is covered by the entirety of an old line}
\label{fig:pusharound2}
\end{figure}

\subsection{Push Beneath} \label{ssec:pbeneath}
The other type of intersect handling that we use is called Push Beneath. Here we do the opposite of push around and handle intersection conflicts by placing the new line, where we want to place it and then moving vertically any old lines that may intersect. If this move creates another intersection we simply move the lines which we pushed into. This is done until no intersections remain. 

In the case where the new line is covered entirely by the old line we avoid intersection as in \cref{fig:pushunderneath1}. By pushing the new line up one level, we intersect the old which needs to move up as well. This warrants that the old line gets support from transport modules.

\begin{figure}[H]
\centering
\includegraphics[width=\textwidth]{conflict3.pdf}
\caption{How push underneath handles the case, where we insert a new line that is covered entirely by an old line}
\label{fig:pushunderneath1}
\end{figure}

In the case where the new line covers the old, we handle intersection as in \cref{fig:pushunderneah2}. As we push up the new line we need to push up the old. However, we need not use transport modules to reconnect the old line, instead it can be reached by flowing vertically through the new line.

Again the cases where there is a partial intersection between old and new line is easy to imagine. If the moving up of an old line creates a new intersection, we just move up the line that was already set in place. This is done until no more intersections occur. When this has been done, some old lines may have been disconnected from their main line. We handle this by allowing items to reach them by moving vertically through existing modules and appending transport modules when needed.  

As can be seen, push underneath functions in a manner opposite to push around. We decide to use it when handling intersections that occur as a result of a parallel transformation. We want our parallel lines to be close to the line, which it sprouted from, otherwise we may not reap the benefits of adding extra modules. This is not needed as much, when doing anti-serialization, which is why we use push around for that instead.

  

\begin{figure}[H]
\centering
\includegraphics[width=\textwidth]{conflict4.pdf}
\caption{How push underneath handles the case, where we insert a new line that  entirely covers an an old line}
\label{fig:pushunderneath2}
\end{figure}

 










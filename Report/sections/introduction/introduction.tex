\chapter{Introduction}
\label{ch:introduction}
In today’s market, manufacturers have to be quick to adapt to new volatile \cref{ch:introduction} trends. Thus, the time window for manufacturing a new product is now very slim, if a profit is to be made. Traditional production lines are not very adaptable to this new market. The items that they are meant to produce are most often decided upon design time. Thus they are not able to cope with the uncertainty of future market requests.

Following this issue, there has been an interest in modular factory systems. This refers to production lines that are made up of individual modules, each with a specific purpose. A module may, for instance, transport items over a conveyor belt, another may hammer bits into place and so on. The modularity of such systems simplifies the process of reconfiguring the production line, when a new item must be produced.

An example a modular factory system is the one developed by FESTO \cite{FESTOweb}, which we will focus on in this report. Manufacturers can order their system from FESTO’s website. This includes an interface, which allows buyers to piece together their production lines before placing an order. This interactive system also indicates, whether a given configuration is valid. A valid configuration is one that can physically be set up and produce the product in question. 

However, manufacturers are not only interested in seeing whether a configuration can function, but also whether it is optimal. Then there is the issue of how to define an optimal configuration, as it may relate to throughput, use of resources, deadlines etc. This is not only an issue when the system is ordered, but also when reconfiguring a production line after initial installation.

During reconfiguration, we need to consider both the physical interfacing between hardware modules as well as software controlling the production line. For workers, the hardware aspects of reconfiguration are rather straightforward, but handling software can be tedious. Each module has a programmable logic controller(PLC), which needs to be reprogrammed to handle different logical cases. The number of cases will grow exponentially as different variants of products need to be considered.  The different modules communicate over a mesh network. Connected to this network is a central hub, used to control production. These parts of the system need also be updated as part of a reconfiguration.

Two overall issues come up when looking at the FESTO system. The first is the need to find an optimal production line configuration, when a given product must be manufactured. The second is the automation of updating software during reconfiguration. We believe the problem of updating software automatically, could be partly solved through the development of a production schedule for a given configuration. From a schedule, PLC code, used to control individual modules, could be generated. These issues are intertwined, as we can choose the optimal configuration from a pool of candidates, by looking at the optimal schedule of each configuration.

We aim to overcome the issues stated, by modeling the FESTO system. By modeling individual configurations, we are able to test and compare these, without having to physically reconfigure an actual production line. We decide to model using timed automata. The temporal properties of this type of model are important to include, as time is a central factor in scheduling. To assist in the modeling process we use the UPPAAL model checker\cite{Larsen97uppaalin}. UPPAAL has been used to model many different real life systems, such as a lacquer production\cite{so54514} and an industrial printer\cite{Igna2008}. From this research we take our inspiration. In these examples however, UPPAAL is used to model systems with only a single configuration. Our work differs in that we try to model a system of which there are many possible configurations. 
